% Options for packages loaded elsewhere
\PassOptionsToPackage{unicode}{hyperref}
\PassOptionsToPackage{hyphens}{url}
%
\documentclass[
]{article}
\usepackage{amsmath,amssymb}
\usepackage{lmodern}
\usepackage{ifxetex,ifluatex}
\ifnum 0\ifxetex 1\fi\ifluatex 1\fi=0 % if pdftex
  \usepackage[T1]{fontenc}
  \usepackage[utf8]{inputenc}
  \usepackage{textcomp} % provide euro and other symbols
\else % if luatex or xetex
  \usepackage{unicode-math}
  \defaultfontfeatures{Scale=MatchLowercase}
  \defaultfontfeatures[\rmfamily]{Ligatures=TeX,Scale=1}
\fi
% Use upquote if available, for straight quotes in verbatim environments
\IfFileExists{upquote.sty}{\usepackage{upquote}}{}
\IfFileExists{microtype.sty}{% use microtype if available
  \usepackage[]{microtype}
  \UseMicrotypeSet[protrusion]{basicmath} % disable protrusion for tt fonts
}{}
\makeatletter
\@ifundefined{KOMAClassName}{% if non-KOMA class
  \IfFileExists{parskip.sty}{%
    \usepackage{parskip}
  }{% else
    \setlength{\parindent}{0pt}
    \setlength{\parskip}{6pt plus 2pt minus 1pt}}
}{% if KOMA class
  \KOMAoptions{parskip=half}}
\makeatother
\usepackage{xcolor}
\IfFileExists{xurl.sty}{\usepackage{xurl}}{} % add URL line breaks if available
\IfFileExists{bookmark.sty}{\usepackage{bookmark}}{\usepackage{hyperref}}
\hypersetup{
  pdftitle={CW - data prep},
  hidelinks,
  pdfcreator={LaTeX via pandoc}}
\urlstyle{same} % disable monospaced font for URLs
\usepackage[margin=1in]{geometry}
\usepackage{color}
\usepackage{fancyvrb}
\newcommand{\VerbBar}{|}
\newcommand{\VERB}{\Verb[commandchars=\\\{\}]}
\DefineVerbatimEnvironment{Highlighting}{Verbatim}{commandchars=\\\{\}}
% Add ',fontsize=\small' for more characters per line
\usepackage{framed}
\definecolor{shadecolor}{RGB}{248,248,248}
\newenvironment{Shaded}{\begin{snugshade}}{\end{snugshade}}
\newcommand{\AlertTok}[1]{\textcolor[rgb]{0.94,0.16,0.16}{#1}}
\newcommand{\AnnotationTok}[1]{\textcolor[rgb]{0.56,0.35,0.01}{\textbf{\textit{#1}}}}
\newcommand{\AttributeTok}[1]{\textcolor[rgb]{0.77,0.63,0.00}{#1}}
\newcommand{\BaseNTok}[1]{\textcolor[rgb]{0.00,0.00,0.81}{#1}}
\newcommand{\BuiltInTok}[1]{#1}
\newcommand{\CharTok}[1]{\textcolor[rgb]{0.31,0.60,0.02}{#1}}
\newcommand{\CommentTok}[1]{\textcolor[rgb]{0.56,0.35,0.01}{\textit{#1}}}
\newcommand{\CommentVarTok}[1]{\textcolor[rgb]{0.56,0.35,0.01}{\textbf{\textit{#1}}}}
\newcommand{\ConstantTok}[1]{\textcolor[rgb]{0.00,0.00,0.00}{#1}}
\newcommand{\ControlFlowTok}[1]{\textcolor[rgb]{0.13,0.29,0.53}{\textbf{#1}}}
\newcommand{\DataTypeTok}[1]{\textcolor[rgb]{0.13,0.29,0.53}{#1}}
\newcommand{\DecValTok}[1]{\textcolor[rgb]{0.00,0.00,0.81}{#1}}
\newcommand{\DocumentationTok}[1]{\textcolor[rgb]{0.56,0.35,0.01}{\textbf{\textit{#1}}}}
\newcommand{\ErrorTok}[1]{\textcolor[rgb]{0.64,0.00,0.00}{\textbf{#1}}}
\newcommand{\ExtensionTok}[1]{#1}
\newcommand{\FloatTok}[1]{\textcolor[rgb]{0.00,0.00,0.81}{#1}}
\newcommand{\FunctionTok}[1]{\textcolor[rgb]{0.00,0.00,0.00}{#1}}
\newcommand{\ImportTok}[1]{#1}
\newcommand{\InformationTok}[1]{\textcolor[rgb]{0.56,0.35,0.01}{\textbf{\textit{#1}}}}
\newcommand{\KeywordTok}[1]{\textcolor[rgb]{0.13,0.29,0.53}{\textbf{#1}}}
\newcommand{\NormalTok}[1]{#1}
\newcommand{\OperatorTok}[1]{\textcolor[rgb]{0.81,0.36,0.00}{\textbf{#1}}}
\newcommand{\OtherTok}[1]{\textcolor[rgb]{0.56,0.35,0.01}{#1}}
\newcommand{\PreprocessorTok}[1]{\textcolor[rgb]{0.56,0.35,0.01}{\textit{#1}}}
\newcommand{\RegionMarkerTok}[1]{#1}
\newcommand{\SpecialCharTok}[1]{\textcolor[rgb]{0.00,0.00,0.00}{#1}}
\newcommand{\SpecialStringTok}[1]{\textcolor[rgb]{0.31,0.60,0.02}{#1}}
\newcommand{\StringTok}[1]{\textcolor[rgb]{0.31,0.60,0.02}{#1}}
\newcommand{\VariableTok}[1]{\textcolor[rgb]{0.00,0.00,0.00}{#1}}
\newcommand{\VerbatimStringTok}[1]{\textcolor[rgb]{0.31,0.60,0.02}{#1}}
\newcommand{\WarningTok}[1]{\textcolor[rgb]{0.56,0.35,0.01}{\textbf{\textit{#1}}}}
\usepackage{graphicx}
\makeatletter
\def\maxwidth{\ifdim\Gin@nat@width>\linewidth\linewidth\else\Gin@nat@width\fi}
\def\maxheight{\ifdim\Gin@nat@height>\textheight\textheight\else\Gin@nat@height\fi}
\makeatother
% Scale images if necessary, so that they will not overflow the page
% margins by default, and it is still possible to overwrite the defaults
% using explicit options in \includegraphics[width, height, ...]{}
\setkeys{Gin}{width=\maxwidth,height=\maxheight,keepaspectratio}
% Set default figure placement to htbp
\makeatletter
\def\fps@figure{htbp}
\makeatother
\setlength{\emergencystretch}{3em} % prevent overfull lines
\providecommand{\tightlist}{%
  \setlength{\itemsep}{0pt}\setlength{\parskip}{0pt}}
\setcounter{secnumdepth}{-\maxdimen} % remove section numbering
\ifluatex
  \usepackage{selnolig}  % disable illegal ligatures
\fi

\title{CW - data prep}
\author{}
\date{\vspace{-2.5em}}

\begin{document}
\maketitle

In this document, you will learn how to process a data layer, and save
it in \texttt{layers} folder.

We'll also walk through the general steps we recommend you include in
your own prep document, including:

\begin{itemize}
\tightlist
\item
  introducing the goal/subgoal, goal model, and data
\item
  setting up directories and loading libraries
\item
  loading and formatting data
\item
  plotting
\item
  saving as .csv in layers folder
\end{itemize}

\emph{The following example was modified from the prep document for
\href{https://github.com/OHI-Science/bhi/blob/draft/baltic2015/prep/CW/eutrophication/eutrophication_prep.Rmd}{OHI-Baltic
Clean Water goal}. We have secchi depth data, which measures water
clarity. The desired data layer will provide average summer secchi depth
in each year and in each region. At the moment, we don't know what the
raw data looks like. We may need to aggregate and manipulate large
quantities of data to get the final, clean data layer. }

\hypertarget{introduction}{%
\section{Introduction}\label{introduction}}

\emph{In this section you'll describe the goal/subgoal, what types of
information or data are needed, data sources, goal model, and how to
approach trend calculation, etc.}

\emph{For example, you can start with a general introduction of what
this goal/subgoal is trying to measure, what it means in your local
context, and what parameters make sense to be included or explored
here.}

This subgoal aims to represent the eutrophication level in the water in
each region. We uses summer time water clarity, measured by secchi
depth, as a proxy indicator, assuming a linear relationship between
water clarity and nutrient levels. More info on secchi depth can be
found
\href{http://www.helcom.fi/baltic-sea-trends/indicators/water-clarity}{here}.

\hypertarget{goal-model}{%
\subsection{Goal model}\label{goal-model}}

\emph{Record what the goal model and reference point should be, how to
approach trend calculations, etc.}

\hypertarget{status}{%
\subsubsection{Status}\label{status}}

Xao = Mean Stock Indicator Value / Reference pt

Stock indicators = two HELCOM core indicators assessed for good
environmental status (each scored between 0 and 1 by BHI)

Reference pt = maximum possible good environmental status (value=1)

\hypertarget{trend}{%
\subsubsection{Trend}\label{trend}}

\emph{Typically we calculate trend as a linear trend of the last five
years of status. In this assessment, however, this approach is not
feasible. And an alternative approach is used and documented here.}

CPUE time series are available for all stations used for the HELCOM
coastal fish populations core indicators. These data were provided by
Jens Olsson (FISH PRO II project). To calculate GES status, full time
series were used. Therefore, only one status time point and cannot
calculate trend of status over time. Instead, we'll follow approach from
Bergström et al 2016, but only focus on the final time period for the
slope (2004-2013).

Bergstrom et al.~2016. Long term changes in the status of coastal fish
in the Baltic Sea. \emph{Estuarin, Coast and Shelf Science}. 169:74-84

Method:

\begin{enumerate}
\def\labelenumi{\arabic{enumi}.}
\item
  Select final time period of trend assessment (2004-2013)
\item
  Use time series from both indicators, Key Species and Functional
  groups. For functional groups,include both cyprinid and piscivore time
  series
\item
  For each time series: square-root transform data, z-score, fit linear
  regression, extract slope
\item
  Within each time series group (key species, cyprinid, piscivore), take
  the mean slope for each group within each basin
\item
  Within each basin take a mean functional group indicator slope (mean
  of cyprinid mean and piscivore mean)
\item
  For each basin take overall mean slope - mean of key species and
  functional group
\item
  Apply trend value for basin to all BHI regions (except in Gulf of
  Finland, do not apply Finnish site value to Estonia and Russian
  regions.)
\end{enumerate}

\hypertarget{data-sources}{%
\subsection{Data sources}\label{data-sources}}

\emph{Here you can record where the data comes from, where it's stored,
potential concerns with the data, why you included or excluded certain
data, etc:}

\textbf{ICE}: Data extracted from database and sent by Hjalte Parner on
Feb 10 2016.

\emph{Note from Parner: ``extraction from our database classified into
HELCOM Assessment Units -- HELCOM sub basins with coastal WFD water
bodies or water types''}

Pros and cons of using these data:

\begin{itemize}
\tightlist
\item
  Pros: these are the most recent published data and thus reflect the
  most current conditions of water quality\ldots{}
\item
  Cons: these datasets don't have full spatial coverage, or don't have
  even temporal coverage, and thus for some regions we need to do gap
  filling\ldots{}
\end{itemize}

Reasons for excluding certain datasets:

Direct measurements of nutrient levels (eg. phosphate, nitrate, etc)
were excluded from this subgoal because not every region measure these
chemicals regularly, or we didn't have time-series data on nutrients to
be able to calculate trend\ldots{}

\hypertarget{data-prep-process}{%
\section{Data prep process}\label{data-prep-process}}

\hypertarget{setup}{%
\subsection{Setup}\label{setup}}

This section will set up directories, functions, call commonly used
libraries, etc, to prepare for the next steps of data prep.

\hypertarget{read-in-data-and-initial-exploration}{%
\subsection{Read in data and initial
exploration}\label{read-in-data-and-initial-exploration}}

ICES and SMHI data have non-overlapping observations and were combined
to one data set for our use. Some observations were not assigned to any
regions (ie. no region IDs attached) and were omitted.

Both data sets contains profile data (eg temperature, but secchi is only
measured once). We need only unique secchi records. Duplicates were thus
taken out.

\emph{Basic functions we will encounter and you will use often: select,
mutate, filter, rename}

\begin{verbatim}
## New names:
## * `` -> ...1
\end{verbatim}

\begin{verbatim}
## Rows: 215543 Columns: 13
\end{verbatim}

\begin{verbatim}
## -- Column specification --------------------------------------------------------
## Delimiter: ","
## chr  (4): Assessment_unit, HELCOM_ID, Cruise, Station
## dbl  (8): ...1, secchi, BHI_ID, Month, Year, HELCOM_COASTAL_CODE, Latitude, ...
## dttm (1): Date
\end{verbatim}

\begin{verbatim}
## 
## i Use `spec()` to retrieve the full column specification for this data.
## i Specify the column types or set `show_col_types = FALSE` to quiet this message.
\end{verbatim}

\begin{verbatim}
## # A tibble: 6 x 13
##    ...1 secchi BHI_ID Month  Year Assessment_unit HELCOM_COASTAL_CODE HELCOM_ID
##   <dbl>  <dbl>  <dbl> <dbl> <dbl> <chr>                         <dbl> <chr>    
## 1     1    2        3     6  1972 DEN-012                          38 SEA-002  
## 2     2    3.5      3     8  1972 DEN-012                          38 SEA-002  
## 3     3    3.5      3     8  1972 DEN-012                          38 SEA-002  
## 4     4    4.5      3    12  1972 DEN-012                          38 SEA-002  
## 5     5    4.5      3    12  1972 DEN-012                          38 SEA-002  
## 6     6    3.1      3     4  1973 DEN-012                          38 SEA-002  
## # ... with 5 more variables: Date <dttm>, Latitude <dbl>, Longitude <dbl>,
## #   Cruise <chr>, Station <chr>
\end{verbatim}

\begin{verbatim}
## spec_tbl_df [215,543 x 13] (S3: spec_tbl_df/tbl_df/tbl/data.frame)
##  $ ...1               : num [1:215543] 1 2 3 4 5 6 7 8 9 10 ...
##  $ secchi             : num [1:215543] 2 3.5 3.5 4.5 4.5 3.1 3.1 1.8 1.8 2.5 ...
##  $ BHI_ID             : num [1:215543] 3 3 3 3 3 3 3 3 3 3 ...
##  $ Month              : num [1:215543] 6 8 8 12 12 4 4 6 6 8 ...
##  $ Year               : num [1:215543] 1972 1972 1972 1972 1972 ...
##  $ Assessment_unit    : chr [1:215543] "DEN-012" "DEN-012" "DEN-012" "DEN-012" ...
##  $ HELCOM_COASTAL_CODE: num [1:215543] 38 38 38 38 38 38 38 38 38 38 ...
##  $ HELCOM_ID          : chr [1:215543] "SEA-002" "SEA-002" "SEA-002" "SEA-002" ...
##  $ Date               : POSIXct[1:215543], format: "1972-06-06 12:00:00" "1972-08-01 12:00:00" ...
##  $ Latitude           : num [1:215543] 55.9 55.9 55.9 55.9 55.9 ...
##  $ Longitude          : num [1:215543] 9.91 9.91 9.91 9.91 9.91 ...
##  $ Cruise             : chr [1:215543] "26ve" "26ve" "26ve" "26ve" ...
##  $ Station            : chr [1:215543] "0008" "0009" "0009" "0010" ...
##  - attr(*, "spec")=
##   .. cols(
##   ..   ...1 = col_double(),
##   ..   secchi = col_double(),
##   ..   BHI_ID = col_double(),
##   ..   Month = col_double(),
##   ..   Year = col_double(),
##   ..   Assessment_unit = col_character(),
##   ..   HELCOM_COASTAL_CODE = col_double(),
##   ..   HELCOM_ID = col_character(),
##   ..   Date = col_datetime(format = ""),
##   ..   Latitude = col_double(),
##   ..   Longitude = col_double(),
##   ..   Cruise = col_character(),
##   ..   Station = col_character()
##   .. )
##  - attr(*, "problems")=<externalptr>
\end{verbatim}

\begin{verbatim}
##   bhi_id secchi year month     lat    lon       date
## 1      3    2.0 1972     6 55.8517 9.9083 1972-06-06
## 2      3    3.5 1972     8 55.8517 9.9083 1972-08-01
## 3      3    3.5 1972     8 55.8517 9.9083 1972-08-01
## 4      3    4.5 1972    12 55.8517 9.9083 1972-12-04
## 5      3    4.5 1972    12 55.8517 9.9083 1972-12-04
## 6      3    3.1 1973     4 55.8517 9.9083 1973-04-12
\end{verbatim}

\begin{verbatim}
## 'data.frame':    215543 obs. of  7 variables:
##  $ bhi_id: num  3 3 3 3 3 3 3 3 3 3 ...
##  $ secchi: num  2 3.5 3.5 4.5 4.5 3.1 3.1 1.8 1.8 2.5 ...
##  $ year  : num  1972 1972 1972 1972 1972 ...
##  $ month : num  6 8 8 12 12 4 4 6 6 8 ...
##  $ lat   : num  55.9 55.9 55.9 55.9 55.9 ...
##  $ lon   : num  9.91 9.91 9.91 9.91 9.91 ...
##  $ date  : Date, format: "1972-06-06" "1972-08-01" ...
\end{verbatim}

\begin{verbatim}
## [1] 1684
\end{verbatim}

\begin{verbatim}
## [1] 180963
\end{verbatim}

\begin{verbatim}
## [1] 32896
\end{verbatim}

\hypertarget{select-only-summer-observations}{%
\subsection{Select only summer
observations}\label{select-only-summer-observations}}

Only summer months post year 2000 were relevant to our use. Therefore we
filtered for data in:

\begin{itemize}
\tightlist
\item
  Months 6-9 (June, July, August, September)\\
\item
  Years 2010-2015
\end{itemize}

The plots showed that some regions don't have good data coverage. Some
basins are missing data for most recent years, such as regions 22 and
25. It appeared that water quality data makes more sense at the basin
level, and will be aggregated to the basin level in the next section.
(This is an uncommon case in the Baltic's case, but it's left here for
demonstration. )

\emph{Plotting is a good way to spot data gaps and other potential
problems. Without plotting, we might have missed these temporal gaps.}

\begin{Shaded}
\begin{Highlighting}[]
\DocumentationTok{\#\# select summer months}
\NormalTok{summer }\OtherTok{=}\NormalTok{ new\_ices }\SpecialCharTok{\%\textgreater{}\%} \FunctionTok{filter}\NormalTok{(month }\SpecialCharTok{\%in\%}\FunctionTok{c}\NormalTok{(}\DecValTok{6}\SpecialCharTok{:}\DecValTok{9}\NormalTok{)) }\SpecialCharTok{\%\textgreater{}\%}
        \FunctionTok{filter}\NormalTok{(year }\SpecialCharTok{\%in\%} \FunctionTok{c}\NormalTok{(}\DecValTok{2010}\SpecialCharTok{:}\DecValTok{2015}\NormalTok{))}
\FunctionTok{head}\NormalTok{(summer)}
\end{Highlighting}
\end{Shaded}

\begin{verbatim}
##   bhi_id secchi year month     lat     lon       date
## 1     24    9.5 2012     6 56.3113 20.8297 2012-06-06
## 2     24    9.3 2012     6 56.4040 19.3545 2012-06-06
## 3     24    6.9 2012     6 56.3702 19.9957 2012-06-07
## 4     24    8.6 2012     6 56.7158 20.4372 2012-06-07
## 5     24    8.1 2012     6 57.3553 20.7755 2012-06-08
## 6     24    7.1 2012     6 57.4107 21.3267 2012-06-08
\end{verbatim}

\begin{Shaded}
\begin{Highlighting}[]
\DocumentationTok{\#\# plot: by month}
\FunctionTok{ggplot}\NormalTok{(summer) }\SpecialCharTok{+} \FunctionTok{geom\_point}\NormalTok{(}\FunctionTok{aes}\NormalTok{(month, secchi))}\SpecialCharTok{+}
  \FunctionTok{facet\_wrap}\NormalTok{(}\SpecialCharTok{\textasciitilde{}}\NormalTok{bhi\_id, }\AttributeTok{scales =}\StringTok{"free\_y"}\NormalTok{)}
\end{Highlighting}
\end{Shaded}

\includegraphics{CW_data_prep_files/figure-latex/select summer data-1.pdf}

\begin{Shaded}
\begin{Highlighting}[]
\DocumentationTok{\#\# plot: by year}
\FunctionTok{ggplot}\NormalTok{(summer) }\SpecialCharTok{+} \FunctionTok{geom\_point}\NormalTok{(}\FunctionTok{aes}\NormalTok{(year,secchi))}\SpecialCharTok{+}
  \FunctionTok{facet\_wrap}\NormalTok{(}\SpecialCharTok{\textasciitilde{}}\NormalTok{bhi\_id)}
\end{Highlighting}
\end{Shaded}

\includegraphics{CW_data_prep_files/figure-latex/select summer data-2.pdf}

\hypertarget{calculate-mean-summer-secchi-depth-by-region}{%
\subsection{Calculate mean summer secchi depth by
region}\label{calculate-mean-summer-secchi-depth-by-region}}

Here we calculated the mean summer secchi depth per year for each
region.

\emph{Plotting per region by year showed that not all regions have
continuous and adequate data. There are differents ways to deal with
that, which we will not explore today. You could read more on gapfilling
in our \href{http://ohi-science.org/manual/\#gapfilling}{OHI Manual}.}

\begin{Shaded}
\begin{Highlighting}[]
\DocumentationTok{\#\# calculate monthly means for each month}
\NormalTok{mean\_months }\OtherTok{=}\NormalTok{ summer }\SpecialCharTok{\%\textgreater{}\%} \FunctionTok{select}\NormalTok{(bhi\_id, year, month, secchi) }\SpecialCharTok{\%\textgreater{}\%}
              \FunctionTok{group\_by}\NormalTok{(bhi\_id, year, month) }\SpecialCharTok{\%\textgreater{}\%}
              \FunctionTok{summarise}\NormalTok{(}\AttributeTok{mean\_secchi =} \FunctionTok{round}\NormalTok{(}\FunctionTok{mean}\NormalTok{(secchi, }\AttributeTok{na.rm=}\ConstantTok{TRUE}\NormalTok{), }\DecValTok{1}\NormalTok{)) }\SpecialCharTok{\%\textgreater{}\%}
              \FunctionTok{ungroup}\NormalTok{()}
\end{Highlighting}
\end{Shaded}

\begin{verbatim}
## `summarise()` has grouped output by 'bhi_id', 'year'. You can override using the `.groups` argument.
\end{verbatim}

\begin{Shaded}
\begin{Highlighting}[]
\FunctionTok{head}\NormalTok{(mean\_months)}
\end{Highlighting}
\end{Shaded}

\begin{verbatim}
## # A tibble: 6 x 4
##   bhi_id  year month mean_secchi
##    <dbl> <dbl> <dbl>       <dbl>
## 1      1  2010     6         7.4
## 2      1  2010     7         8.8
## 3      1  2010     8         9.5
## 4      1  2010     9         7.7
## 5      1  2011     6         6  
## 6      1  2011     7         7.7
\end{verbatim}

\begin{Shaded}
\begin{Highlighting}[]
\DocumentationTok{\#\# plot monthly means }
\FunctionTok{ggplot}\NormalTok{(mean\_months) }\SpecialCharTok{+} \FunctionTok{geom\_point}\NormalTok{(}\FunctionTok{aes}\NormalTok{(year,mean\_secchi, }\AttributeTok{colour=}\FunctionTok{factor}\NormalTok{(month))) }\SpecialCharTok{+}
  \FunctionTok{geom\_line}\NormalTok{(}\FunctionTok{aes}\NormalTok{(year, mean\_secchi, }\AttributeTok{colour=}\FunctionTok{factor}\NormalTok{(month))) }\SpecialCharTok{+}
  \FunctionTok{facet\_wrap}\NormalTok{(}\SpecialCharTok{\textasciitilde{}}\NormalTok{bhi\_id)}\SpecialCharTok{+}
  \FunctionTok{scale\_y\_continuous}\NormalTok{(}\AttributeTok{limits =} \FunctionTok{c}\NormalTok{(}\DecValTok{0}\NormalTok{,}\DecValTok{10}\NormalTok{))}
\end{Highlighting}
\end{Shaded}

\begin{verbatim}
## Warning: Removed 4 rows containing missing values (geom_point).
\end{verbatim}

\begin{verbatim}
## geom_path: Each group consists of only one observation. Do you need to adjust
## the group aesthetic?
\end{verbatim}

\begin{verbatim}
## geom_path: Each group consists of only one observation. Do you need to adjust
## the group aesthetic?
## geom_path: Each group consists of only one observation. Do you need to adjust
## the group aesthetic?
## geom_path: Each group consists of only one observation. Do you need to adjust
## the group aesthetic?
## geom_path: Each group consists of only one observation. Do you need to adjust
## the group aesthetic?
## geom_path: Each group consists of only one observation. Do you need to adjust
## the group aesthetic?
\end{verbatim}

\includegraphics{CW_data_prep_files/figure-latex/mean monthly secchi depth-1.pdf}

\begin{Shaded}
\begin{Highlighting}[]
\DocumentationTok{\#\# calculate summer means by region}
\DocumentationTok{\#\# region summer means = mean of region monthly mean values}

\NormalTok{mean\_months\_summer }\OtherTok{=}\NormalTok{ mean\_months }\SpecialCharTok{\%\textgreater{}\%} 
                      \FunctionTok{group\_by}\NormalTok{(bhi\_id, year)}\SpecialCharTok{\%\textgreater{}\%}
                      \FunctionTok{summarise}\NormalTok{(}\AttributeTok{mean\_secchi =} \FunctionTok{round}\NormalTok{(}\FunctionTok{mean}\NormalTok{(mean\_secchi, }\AttributeTok{na.rm=}\ConstantTok{TRUE}\NormalTok{), }\DecValTok{1}\NormalTok{)) }\SpecialCharTok{\%\textgreater{}\%}
                      \FunctionTok{ungroup}\NormalTok{()  }\SpecialCharTok{\%\textgreater{}\%} 
  \FunctionTok{rename}\NormalTok{(}\AttributeTok{rgn\_id =}\NormalTok{ bhi\_id) }\DocumentationTok{\#\#\#\# rgn\_id is a required field in a data layer!! }\AlertTok{\#\#\#}
\end{Highlighting}
\end{Shaded}

\begin{verbatim}
## `summarise()` has grouped output by 'bhi_id'. You can override using the `.groups` argument.
\end{verbatim}

\begin{Shaded}
\begin{Highlighting}[]
\DocumentationTok{\#\# plot summer means by basin}
\FunctionTok{ggplot}\NormalTok{(mean\_months\_summer) }\SpecialCharTok{+} 
  \FunctionTok{geom\_point}\NormalTok{(}\FunctionTok{aes}\NormalTok{(year,mean\_secchi)) }\SpecialCharTok{+}
  \FunctionTok{geom\_line}\NormalTok{(}\FunctionTok{aes}\NormalTok{(year,mean\_secchi))}\SpecialCharTok{+}
  \FunctionTok{facet\_wrap}\NormalTok{(}\SpecialCharTok{\textasciitilde{}}\NormalTok{rgn\_id)}\SpecialCharTok{+}
  \FunctionTok{scale\_y\_continuous}\NormalTok{(}\AttributeTok{limits =} \FunctionTok{c}\NormalTok{(}\DecValTok{0}\NormalTok{,}\DecValTok{10}\NormalTok{))}
\end{Highlighting}
\end{Shaded}

\begin{verbatim}
## Warning: Removed 1 rows containing missing values (geom_point).
\end{verbatim}

\begin{verbatim}
## geom_path: Each group consists of only one observation. Do you need to adjust
## the group aesthetic?
## geom_path: Each group consists of only one observation. Do you need to adjust
## the group aesthetic?
## geom_path: Each group consists of only one observation. Do you need to adjust
## the group aesthetic?
## geom_path: Each group consists of only one observation. Do you need to adjust
## the group aesthetic?
## geom_path: Each group consists of only one observation. Do you need to adjust
## the group aesthetic?
## geom_path: Each group consists of only one observation. Do you need to adjust
## the group aesthetic?
\end{verbatim}

\includegraphics{CW_data_prep_files/figure-latex/mean monthly secchi depth-2.pdf}

\emph{Congratulations! You have successfully wrangled a large dataset
and produced a clean data layer. In order for the Toolbox to use it for
calculation, we need to save it in the right location and register it.
Let's go back to the Training page for a minute and learn from there.}

\hypertarget{save-data-layer-to-layers-folder}{%
\subsection{\texorpdfstring{Save data layer to \texttt{layers}
folder}{Save data layer to layers folder}}\label{save-data-layer-to-layers-folder}}

When modifying existing or creating new data layers in the prep folder,
save the ready-to-use layers in \texttt{layers} folder. We recommend
saving it as a new \emph{.csv} file with:

\begin{itemize}
\tightlist
\item
  a prefix identifying the goal (eg. \texttt{fis\_})
\item
  a suffix identifying your assessment (eg. \texttt{\_sc2016.csv}).
\end{itemize}

Modifying the layer name provides an easy way to track which data layers
have been updated regionally, and which rely on global data. Then, the
original layers (\texttt{\_gl2016.csv} and
\texttt{\_gl2016placeholder.csv}) can be deleted.

\emph{\textbf{Tip}: filenames should not have any spaces: use an
underscore instead. This will reduce problems when R reads the files.}

\begin{Shaded}
\begin{Highlighting}[]
\FunctionTok{write\_csv}\NormalTok{(mean\_months\_summer, }\FunctionTok{file.path}\NormalTok{(dir\_layers, }\StringTok{"cw\_mean\_secchi\_region2016\_1.csv"}\NormalTok{))}
\end{Highlighting}
\end{Shaded}

\textbf{Last Step: To render this file and use it to communicate with
other human beings}:

Click on \emph{Knit} at the top of this window to create an .md file,
which can be displayed online as a webpage. Pull/Commit/Push as usual
for this file. Click on the .md file in your repo on Github.com to view
the rendered file.

\emph{Now we have saved the layer in \texttt{layers} folder, there is
only one more thing left to do - register it in the data registry. Let's
switch back to where we left off in Training.}

\end{document}
